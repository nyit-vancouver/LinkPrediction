\section{Conclusions and Future Work}

We have studied the problem of relationship prediction in DHINs and proposed a supervised learning framework based on a combined set of latent and topological meta path-based features. Our results show that the proposed technique significantly improves prediction accuracy compared to the baseline methods. As a part of future work and given the major computational bottleneck of methods that rely on meta-paths, such as our approach, is calculating meta path-based measures, we would like to investigate approximation techniques to make  the prediction process scalable. Furthermore, we are interested in enhancing the matrix factorization technique based on a loss function that does not require the full topological features matrix. %In addition to model improvement, %
Another interesting direction to investigate is the effectiveness of our proposed approach in other application domains such as predicting user interests in a social network that is both temporally dynamic and heterogeneous by nature. Link prediction techniques may also increase the risk of link disclosure, such as through link reconstruction and re-identification attacks \cite{zheleva2008preserving,fire2013links}, and thus increase privacy concern. It is interesting to study the effect of our technique in performance of link privacy preserving methods, such as \cite{hay2008resisting,zheleva2008preserving,amin:edbt12,amin:wwwj}, and propose suggestions for improvement.


%Michael et al. \cite{fire2013links} presented a link reconstruction attack, in which the attacker uses link prediction to infer a user's connections to others with high accuracy, but they did not mention how to defend the so-called link-reconstruction attack. Since link-reconstruction attack or Link prediction-based attack aims to find out some real but unobservable links, the defense of link-prediction-based attacks is also target-directed, which means that one has to preserve the targeted links from being predicted. Most existing approaches on link prediction are based on the similarity between nodes under the assumption that the more similar a pair of nodes are, the more likely a link exists between them.

%Zheleva et al. \cite{zheleva2008preserving} proposed the concept of link re-identification attack, which refers to inferring sensitive relationships from anonymized network data. If the sensitive links can be identified by the released data, then this means privacy breach. Link perturbation is a common technique to preserve sensitive links. 
%\cite{zheleva2008preserving,ying2008randomizing} investigated the relationship between the level of link randomization and the possibility to infer the presence of a link in a network. Further, Ying et al. \cite{ying2009link} investigated the effect of link randomization on protecting privacy of sensitive links, and they found that similarity indices can be utilized by adversaries to significantly improve the accuracy in predicting sensitive links.




%we also observe that network characteristics can impact prediction accuracy. For instance, as shown in Figure ......



%We studied the problem of relationship prediction in DHINs and proposed a supervised learning framework based on a combined set of latent and topological meta path-based features. Our results show that the proposed technique significantly improves prediction accuracy compared to the baseline methods. In this work we did not evaluate the running time and efficiency of our approach. Since our major computational bottleneck is calculating meta path-based measures, such as path count, we would like to investigate approximation techniques to make the prediction scalable. Furthermore we are interested in enhancing the matrix factorization technique based on a loss function that does not require full topological features matrix. In addition to model improvement, another interesting direction is to investigate the effectiveness of our proposed approach in other applications, such as predicting interests of users in a social media, that can be formulated as a link/relationship prediction problem.